% Options for packages loaded elsewhere
\PassOptionsToPackage{unicode}{hyperref}
\PassOptionsToPackage{hyphens}{url}
\PassOptionsToPackage{dvipsnames,svgnames,x11names}{xcolor}
%
\documentclass[
  a4paper,
  DIV=11,
  numbers=noendperiod]{scrartcl}

\usepackage{amsmath,amssymb}
\usepackage{lmodern}
\usepackage{iftex}
\ifPDFTeX
  \usepackage[T1]{fontenc}
  \usepackage[utf8]{inputenc}
  \usepackage{textcomp} % provide euro and other symbols
\else % if luatex or xetex
  \usepackage{unicode-math}
  \defaultfontfeatures{Scale=MatchLowercase}
  \defaultfontfeatures[\rmfamily]{Ligatures=TeX,Scale=1}
  \setmainfont[]{Tex Gyre Termes}
\fi
% Use upquote if available, for straight quotes in verbatim environments
\IfFileExists{upquote.sty}{\usepackage{upquote}}{}
\IfFileExists{microtype.sty}{% use microtype if available
  \usepackage[]{microtype}
  \UseMicrotypeSet[protrusion]{basicmath} % disable protrusion for tt fonts
}{}
\makeatletter
\@ifundefined{KOMAClassName}{% if non-KOMA class
  \IfFileExists{parskip.sty}{%
    \usepackage{parskip}
  }{% else
    \setlength{\parindent}{0pt}
    \setlength{\parskip}{6pt plus 2pt minus 1pt}}
}{% if KOMA class
  \KOMAoptions{parskip=half}}
\makeatother
\usepackage{xcolor}
\usepackage[top=30mm,left=15mm,right = 15mm,heightrounded]{geometry}
\setlength{\emergencystretch}{3em} % prevent overfull lines
\setcounter{secnumdepth}{5}
% Make \paragraph and \subparagraph free-standing
\ifx\paragraph\undefined\else
  \let\oldparagraph\paragraph
  \renewcommand{\paragraph}[1]{\oldparagraph{#1}\mbox{}}
\fi
\ifx\subparagraph\undefined\else
  \let\oldsubparagraph\subparagraph
  \renewcommand{\subparagraph}[1]{\oldsubparagraph{#1}\mbox{}}
\fi

\usepackage{color}
\usepackage{fancyvrb}
\newcommand{\VerbBar}{|}
\newcommand{\VERB}{\Verb[commandchars=\\\{\}]}
\DefineVerbatimEnvironment{Highlighting}{Verbatim}{commandchars=\\\{\}}
% Add ',fontsize=\small' for more characters per line
\usepackage{framed}
\definecolor{shadecolor}{RGB}{241,243,245}
\newenvironment{Shaded}{\begin{snugshade}}{\end{snugshade}}
\newcommand{\AlertTok}[1]{\textcolor[rgb]{0.68,0.00,0.00}{#1}}
\newcommand{\AnnotationTok}[1]{\textcolor[rgb]{0.37,0.37,0.37}{#1}}
\newcommand{\AttributeTok}[1]{\textcolor[rgb]{0.40,0.45,0.13}{#1}}
\newcommand{\BaseNTok}[1]{\textcolor[rgb]{0.68,0.00,0.00}{#1}}
\newcommand{\BuiltInTok}[1]{\textcolor[rgb]{0.00,0.23,0.31}{#1}}
\newcommand{\CharTok}[1]{\textcolor[rgb]{0.13,0.47,0.30}{#1}}
\newcommand{\CommentTok}[1]{\textcolor[rgb]{0.37,0.37,0.37}{#1}}
\newcommand{\CommentVarTok}[1]{\textcolor[rgb]{0.37,0.37,0.37}{\textit{#1}}}
\newcommand{\ConstantTok}[1]{\textcolor[rgb]{0.56,0.35,0.01}{#1}}
\newcommand{\ControlFlowTok}[1]{\textcolor[rgb]{0.00,0.23,0.31}{#1}}
\newcommand{\DataTypeTok}[1]{\textcolor[rgb]{0.68,0.00,0.00}{#1}}
\newcommand{\DecValTok}[1]{\textcolor[rgb]{0.68,0.00,0.00}{#1}}
\newcommand{\DocumentationTok}[1]{\textcolor[rgb]{0.37,0.37,0.37}{\textit{#1}}}
\newcommand{\ErrorTok}[1]{\textcolor[rgb]{0.68,0.00,0.00}{#1}}
\newcommand{\ExtensionTok}[1]{\textcolor[rgb]{0.00,0.23,0.31}{#1}}
\newcommand{\FloatTok}[1]{\textcolor[rgb]{0.68,0.00,0.00}{#1}}
\newcommand{\FunctionTok}[1]{\textcolor[rgb]{0.28,0.35,0.67}{#1}}
\newcommand{\ImportTok}[1]{\textcolor[rgb]{0.00,0.46,0.62}{#1}}
\newcommand{\InformationTok}[1]{\textcolor[rgb]{0.37,0.37,0.37}{#1}}
\newcommand{\KeywordTok}[1]{\textcolor[rgb]{0.00,0.23,0.31}{#1}}
\newcommand{\NormalTok}[1]{\textcolor[rgb]{0.00,0.23,0.31}{#1}}
\newcommand{\OperatorTok}[1]{\textcolor[rgb]{0.37,0.37,0.37}{#1}}
\newcommand{\OtherTok}[1]{\textcolor[rgb]{0.00,0.23,0.31}{#1}}
\newcommand{\PreprocessorTok}[1]{\textcolor[rgb]{0.68,0.00,0.00}{#1}}
\newcommand{\RegionMarkerTok}[1]{\textcolor[rgb]{0.00,0.23,0.31}{#1}}
\newcommand{\SpecialCharTok}[1]{\textcolor[rgb]{0.37,0.37,0.37}{#1}}
\newcommand{\SpecialStringTok}[1]{\textcolor[rgb]{0.13,0.47,0.30}{#1}}
\newcommand{\StringTok}[1]{\textcolor[rgb]{0.13,0.47,0.30}{#1}}
\newcommand{\VariableTok}[1]{\textcolor[rgb]{0.07,0.07,0.07}{#1}}
\newcommand{\VerbatimStringTok}[1]{\textcolor[rgb]{0.13,0.47,0.30}{#1}}
\newcommand{\WarningTok}[1]{\textcolor[rgb]{0.37,0.37,0.37}{\textit{#1}}}

\providecommand{\tightlist}{%
  \setlength{\itemsep}{0pt}\setlength{\parskip}{0pt}}\usepackage{longtable,booktabs,array}
\usepackage{calc} % for calculating minipage widths
% Correct order of tables after \paragraph or \subparagraph
\usepackage{etoolbox}
\makeatletter
\patchcmd\longtable{\par}{\if@noskipsec\mbox{}\fi\par}{}{}
\makeatother
% Allow footnotes in longtable head/foot
\IfFileExists{footnotehyper.sty}{\usepackage{footnotehyper}}{\usepackage{footnote}}
\makesavenoteenv{longtable}
\usepackage{graphicx}
\makeatletter
\def\maxwidth{\ifdim\Gin@nat@width>\linewidth\linewidth\else\Gin@nat@width\fi}
\def\maxheight{\ifdim\Gin@nat@height>\textheight\textheight\else\Gin@nat@height\fi}
\makeatother
% Scale images if necessary, so that they will not overflow the page
% margins by default, and it is still possible to overwrite the defaults
% using explicit options in \includegraphics[width, height, ...]{}
\setkeys{Gin}{width=\maxwidth,height=\maxheight,keepaspectratio}
% Set default figure placement to htbp
\makeatletter
\def\fps@figure{htbp}
\makeatother

\KOMAoption{captions}{tableheading}
\makeatletter
\makeatother
\makeatletter
\makeatother
\makeatletter
\@ifpackageloaded{caption}{}{\usepackage{caption}}
\AtBeginDocument{%
\ifdefined\contentsname
  \renewcommand*\contentsname{Table of contents}
\else
  \newcommand\contentsname{Table of contents}
\fi
\ifdefined\listfigurename
  \renewcommand*\listfigurename{List of Figures}
\else
  \newcommand\listfigurename{List of Figures}
\fi
\ifdefined\listtablename
  \renewcommand*\listtablename{List of Tables}
\else
  \newcommand\listtablename{List of Tables}
\fi
\ifdefined\figurename
  \renewcommand*\figurename{Figure}
\else
  \newcommand\figurename{Figure}
\fi
\ifdefined\tablename
  \renewcommand*\tablename{Table}
\else
  \newcommand\tablename{Table}
\fi
}
\@ifpackageloaded{float}{}{\usepackage{float}}
\floatstyle{ruled}
\@ifundefined{c@chapter}{\newfloat{codelisting}{h}{lop}}{\newfloat{codelisting}{h}{lop}[chapter]}
\floatname{codelisting}{Listing}
\newcommand*\listoflistings{\listof{codelisting}{List of Listings}}
\makeatother
\makeatletter
\@ifpackageloaded{caption}{}{\usepackage{caption}}
\@ifpackageloaded{subcaption}{}{\usepackage{subcaption}}
\makeatother
\makeatletter
\@ifpackageloaded{tcolorbox}{}{\usepackage[many]{tcolorbox}}
\makeatother
\makeatletter
\@ifundefined{shadecolor}{\definecolor{shadecolor}{rgb}{.97, .97, .97}}
\makeatother
\makeatletter
\makeatother
\ifLuaTeX
  \usepackage{selnolig}  % disable illegal ligatures
\fi
\IfFileExists{bookmark.sty}{\usepackage{bookmark}}{\usepackage{hyperref}}
\IfFileExists{xurl.sty}{\usepackage{xurl}}{} % add URL line breaks if available
\urlstyle{same} % disable monospaced font for URLs
\hypersetup{
  pdftitle={WorkingWithDataAssignment},
  pdfauthor={João Mota},
  colorlinks=true,
  linkcolor={blue},
  filecolor={Maroon},
  citecolor={Blue},
  urlcolor={Blue},
  pdfcreator={LaTeX via pandoc}}

\title{WorkingWithDataAssignment}
\author{João Mota}
\date{}

\begin{document}
\maketitle
\ifdefined\Shaded\renewenvironment{Shaded}{\begin{tcolorbox}[boxrule=0pt, enhanced, sharp corners, borderline west={3pt}{0pt}{shadecolor}, interior hidden, breakable, frame hidden]}{\end{tcolorbox}}\fi

\renewcommand*\contentsname{Table of contents}
{
\hypersetup{linkcolor=}
\setcounter{tocdepth}{3}
\tableofcontents
}
/*\\
A few hours of trial and errors can save you a few minutes of reading
the proper documentation :)
https://quarto.org/docs/output-formats/pdf-basics.html\\
Go to terminal tab down there and type quarto install tool tinytex\\
NOTE TO SELF!!!! using quarto is the same as playing restart Rstudio
simulator 2022 because nothing is properly recached and they have a
worse garbage collecter than assembly so if you still get the same error
after changing the just restart rstudio and remember to never ever ever
change the initial format or add anything close to it because it will
break the pdf and start generating html also please be smart and read
https://quarto.org/docs/reference/formats/pdf.html for the formating\\
*/

\begin{Shaded}
\begin{Highlighting}[]
\FunctionTok{library}\NormalTok{(haven)}
\FunctionTok{library}\NormalTok{(tidyverse)}
\end{Highlighting}
\end{Shaded}

\begin{verbatim}
-- Attaching packages --------------------------------------- tidyverse 1.3.2 --
v ggplot2 3.3.6      v purrr   0.3.4 
v tibble  3.1.8      v dplyr   1.0.10
v tidyr   1.2.1      v stringr 1.4.1 
v readr   2.1.3      v forcats 0.5.2 
-- Conflicts ------------------------------------------ tidyverse_conflicts() --
x dplyr::filter() masks stats::filter()
x dplyr::lag()    masks stats::lag()
\end{verbatim}

\begin{Shaded}
\begin{Highlighting}[]
\FunctionTok{library}\NormalTok{(dplyr)}
\FunctionTok{library}\NormalTok{(geometry) }
\CommentTok{\#install.packages(\textquotesingle{}hyperref\textquotesingle{})}
\FunctionTok{library}\NormalTok{(formatR)}
\end{Highlighting}
\end{Shaded}

\begin{Shaded}
\begin{Highlighting}[]
\DocumentationTok{\#\# Very important documentation for the 2018 data set //it is a}
\DocumentationTok{\#\# surprise toll that will help us later}
\NormalTok{technicalAnnex2018 }\OtherTok{=} \StringTok{"https://doc.ukdataservice.ac.uk/doc/8406/mrdoc/pdf/8406\_cyber\_security\_breaches\_survey\_2018\_technical\_annex.pdf"}

\DocumentationTok{\#\# this is the loading the first year of this level of survey data set}
\DocumentationTok{\#\# after burning my entire brain, replacing it with the backup one and}
\DocumentationTok{\#\# also burning that one I discovered that it is just these lines that}
\DocumentationTok{\#\# aren\textquotesingle{}t being formatted in pdf because they are absolutely huge but}
\DocumentationTok{\#\# at least it works for the other ones \#FicaADica I assume it was}
\DocumentationTok{\#\# thanks to formatR ?? I won\textquotesingle{}t bother to redo every single bloody step}
\DocumentationTok{\#\# again, enough pain and stack for the day}
\NormalTok{dataCyberSecuritySurvey2018 }\OtherTok{=} \FunctionTok{read\_spss}\NormalTok{(}\StringTok{"C:/AppliedDataScienceAndStatistics/Applied{-}Data{-}Science{-}and{-}Statistics/Term1/WorkingWithData/Assignment1/Cyber Security Breaches Survey, 2018/UKDA{-}8406{-}spss{-}CyberSurvey2018/spss/spss24/17{-}054294{-}01\_csbs\_2018\_final\_data\_deidentified\_v1\_12042018\_public.sav"}\NormalTok{)}

\DocumentationTok{\#\# adding the variable year because none of the data sets have any}
\DocumentationTok{\#\# proper way to distinguish between the years of each survey}
\NormalTok{dataCyberSecuritySurvey2018}\SpecialCharTok{$}\NormalTok{year }\OtherTok{=} \StringTok{"2018"}
\end{Highlighting}
\end{Shaded}

\hypertarget{now-we-do-the-same-for-the-other-years-before-we-merge-them}{%
\section{Now we do the same for the other years before we merge
them}\label{now-we-do-the-same-for-the-other-years-before-we-merge-them}}

\begin{Shaded}
\begin{Highlighting}[]
\DocumentationTok{\#\# loading the second year of this level of survey data set}
\NormalTok{dataCyberSecuritySurvey2019 }\OtherTok{=} \FunctionTok{read\_spss}\NormalTok{(}\StringTok{"C:/AppliedDataScienceAndStatistics/Applied{-}Data{-}Science{-}and{-}Statistics/Term1/WorkingWithData/Assignment1/Cyber Security Breaches Survey, 2019/UKDA{-}8480{-}spss{-}CyberSurvey2019/spss/spss24/csbs\_2019\_raw\_data.sav"}\NormalTok{)}

\DocumentationTok{\#\# adding the variable year because none of the data sets have any}
\DocumentationTok{\#\# proper way to distinguish between the years of each survey}
\NormalTok{dataCyberSecuritySurvey2019}\SpecialCharTok{$}\NormalTok{year }\OtherTok{=} \StringTok{"2019"}
\end{Highlighting}
\end{Shaded}

\begin{Shaded}
\begin{Highlighting}[]
\DocumentationTok{\#\# loading the third year of this level of survey data set}
\NormalTok{dataCyberSecuritySurvey2020 }\OtherTok{=} \FunctionTok{read\_spss}\NormalTok{(}\StringTok{"C:/AppliedDataScienceAndStatistics/Applied{-}Data{-}Science{-}and{-}Statistics/Term1/WorkingWithData/Assignment1/Cyber Security Breaches Survey, 2020/UKDA{-}8638{-}spss{-}CyberSurvey2020/spss/spss25/csbs2020\_final\_banded\_variables\_only\_180320\_public.sav"}\NormalTok{)}

\DocumentationTok{\#\# adding the variable year because none of the data sets have any}
\DocumentationTok{\#\# proper way to distinguish between the years of each survey}
\NormalTok{dataCyberSecuritySurvey2020}\SpecialCharTok{$}\NormalTok{year }\OtherTok{=} \StringTok{"2020"}
\end{Highlighting}
\end{Shaded}

\begin{Shaded}
\begin{Highlighting}[]
\DocumentationTok{\#\# loading the forth year of this level of survey data set}
\NormalTok{dataCyberSecuritySurvey2021 }\OtherTok{=} \FunctionTok{read\_spss}\NormalTok{(}\StringTok{"C:/AppliedDataScienceAndStatistics/Applied{-}Data{-}Science{-}and{-}Statistics/Term1/WorkingWithData/Assignment1/Cyber Security Breaches Survey, 2020/UKDA{-}8638{-}spss{-}CyberSurvey2020/spss/spss25/csbs2020\_final\_banded\_variables\_only\_180320\_public.sav"}\NormalTok{)}

\DocumentationTok{\#\# adding the variable year because none of the data sets have any}
\DocumentationTok{\#\# proper way to distinguish between the years of each survey}
\NormalTok{dataCyberSecuritySurvey2021}\SpecialCharTok{$}\NormalTok{year }\OtherTok{=} \StringTok{"2021"}
\end{Highlighting}
\end{Shaded}

\begin{Shaded}
\begin{Highlighting}[]
\DocumentationTok{\#\# loading the fifth and final year of this level of survey data set}
\NormalTok{dataCyberSecuritySurvey2022 }\OtherTok{=} \FunctionTok{read\_spss}\NormalTok{(}\StringTok{"C:/AppliedDataScienceAndStatistics/Applied{-}Data{-}Science{-}and{-}Statistics/Term1/WorkingWithData/Assignment1/Cyber Security Breaches Survey, 2022/UKDA{-}8970{-}spss/spss/spss25/dcms\_csbs\_2022\_banded\_cost\_data\_only\_v4\_public.sav"}\NormalTok{)}

\DocumentationTok{\#\# adding the variable year because none of the data sets have any}
\DocumentationTok{\#\# proper way to distinguish between the years of each survey}
\NormalTok{dataCyberSecuritySurvey2022}\SpecialCharTok{$}\NormalTok{year }\OtherTok{=} \StringTok{"2022"}
\end{Highlighting}
\end{Shaded}

\begin{Shaded}
\begin{Highlighting}[]
\DocumentationTok{\#\# Now that we have all data loaded lets start by tidying up data set}
\DocumentationTok{\#\# by data set start from 2018}


\DocumentationTok{\#\# for some sweet sweet documentation about the questions starting from}
\DocumentationTok{\#\# page 26 }\AlertTok{TODO}\DocumentationTok{ comment in case of fire or debugging}
\FunctionTok{browseURL}\NormalTok{(technicalAnnex2018)}
\end{Highlighting}
\end{Shaded}

\begin{Shaded}
\begin{Highlighting}[]
\DocumentationTok{\#\# This entire code snippet is tidying up the type of organisation for}
\DocumentationTok{\#\# the 2018 survey renaming the bloody variables to a more java like}
\DocumentationTok{\#\# name}

\NormalTok{dataCyberSecuritySurvey2018TidyName }\OtherTok{=} \FunctionTok{rename}\NormalTok{(dataCyberSecuritySurvey2018,}
    \AttributeTok{instituitionTypes =} \StringTok{"samptype"}\NormalTok{)}

\DocumentationTok{\#\# if instituitionTypes is 1 it is a business if it is 2 it is a}
\DocumentationTok{\#\# charity and in the future 3 is for schools and education}


\DocumentationTok{\#\# daily reminder that there is a boolean type but it is called logical}
\DocumentationTok{\#\# Numeric {-}\textbackslash{}tSet of all real numbers Integer {-}\textbackslash{}tSet of all integers, Z}
\DocumentationTok{\#\# Logical {-} {-}\textbackslash{}tTRUE and FALSE Complex {-}\textbackslash{}tSet of complex numbers}
\DocumentationTok{\#\# Character {-}\textbackslash{}t“a”, “b”, “c”, …, “ç”, “\#”, “\textasciitilde{}”, …., “1”, “2”, …etc}


\DocumentationTok{\#\# it is a string so lets make it a proper numeric code}

\NormalTok{dataCyberSecuritySurvey2018TidyName}\SpecialCharTok{$}\NormalTok{instituitionTypes }\OtherTok{=} \FunctionTok{as.integer}\NormalTok{(dataCyberSecuritySurvey2018TidyName}\SpecialCharTok{$}\NormalTok{instituitionTypes)}


\DocumentationTok{\#\# typex is 1{-}2 for businesses and 3 for charities so redundant and can}
\DocumentationTok{\#\# be removed}

\NormalTok{dataCyberSecuritySurvey2018TidyName }\OtherTok{=}\NormalTok{ dataCyberSecuritySurvey2018TidyName }\SpecialCharTok{\%\textgreater{}\%}
    \FunctionTok{select}\NormalTok{(}\SpecialCharTok{{-}}\NormalTok{typex)}


\DocumentationTok{\#\# dataCyberSecuritySurvey2018TidyName never forget if R can\textquotesingle{}t show all}
\DocumentationTok{\#\# displayed text from a computation it breaks both the rendering and}
\DocumentationTok{\#\# \#\#the refreshing of the rendered code for some reason ¯/\_(ツ)\_/¯}
\DocumentationTok{\#\# future edit anything and everything breaks for no reason at all,}
\DocumentationTok{\#\# just kill it and reopen refer to the first }\AlertTok{NOTE}\DocumentationTok{ TO SELF for more}
\DocumentationTok{\#\# information}
\end{Highlighting}
\end{Shaded}

\begin{Shaded}
\begin{Highlighting}[]
\DocumentationTok{\#\# see questioner documentation start from page 27}

\NormalTok{technicalAnnex2019 }\OtherTok{=} \StringTok{"https://assets.publishing.service.gov.uk/government/uploads/system/uploads/attachment\_data/file/950097/Cyber\_Security\_Breaches\_Survey\_2019\_{-}\_Technical\_Annex\_V2.pdf"}
\DocumentationTok{\#\# }\AlertTok{TODO}\DocumentationTok{ comment in case of fire or debugging}
\FunctionTok{browseURL}\NormalTok{(technicalAnnex2019)}
\end{Highlighting}
\end{Shaded}

\begin{Shaded}
\begin{Highlighting}[]
\DocumentationTok{\#\# see questioner documentation start from page 31}

\NormalTok{technicalAnnex2020 }\OtherTok{=} \StringTok{"https://assets.publishing.service.gov.uk/government/uploads/system/uploads/attachment\_data/file/874693/Technical\_annex\_{-}\_Cyber\_Security\_Breaches\_Survey\_2020.pdf"}

\DocumentationTok{\#\# }\AlertTok{TODO}\DocumentationTok{ comment in case of fire or debugging}
\FunctionTok{browseURL}\NormalTok{(technicalAnnex2020)}
\end{Highlighting}
\end{Shaded}

\begin{Shaded}
\begin{Highlighting}[]
\DocumentationTok{\#\# see questioner documentation start from page 28}

\NormalTok{technicalAnnex2021 }\OtherTok{=} \StringTok{"https://assets.publishing.service.gov.uk/government/uploads/system/uploads/attachment\_data/file/977491/20{-}046099{-}01\_CSBS\_2021\_quant\_technical\_annex\_v2.4\_\_clean\_\_190321.pdf"}

\DocumentationTok{\#\# }\AlertTok{TODO}\DocumentationTok{ comment in case of fire or debugging}
\FunctionTok{browseURL}\NormalTok{(technicalAnnex2021)}
\end{Highlighting}
\end{Shaded}

\begin{Shaded}
\begin{Highlighting}[]
\DocumentationTok{\#\# see questioner documentation start from page 36}

\NormalTok{technicalAnnex2022 }\OtherTok{=} \StringTok{"https://assets.publishing.service.gov.uk/government/uploads/system/uploads/attachment\_data/file/1064446/Technical\_annex\_{-}\_cyber\_security\_breaches\_survey\_March\_2022\_\_WEB\_.pdf"}

\DocumentationTok{\#\# }\AlertTok{TODO}\DocumentationTok{ comment in case of fire or debugging}
\FunctionTok{browseURL}\NormalTok{(technicalAnnex2022)}

\DocumentationTok{\#\# trying not to get arrested for DDoSing the uk goverment by making a}
\DocumentationTok{\#\# request to all the pdfs after rendering the page for the nth because}
\DocumentationTok{\#\# I can\textquotesingle{}t code nor debug (challenge impossible) bonus points if I get}
\DocumentationTok{\#\# an exeter ip banned because of it}
\end{Highlighting}
\end{Shaded}

\begin{Shaded}
\begin{Highlighting}[]
\DocumentationTok{\#\# time to recycle the code for the 2018 survey that gets a \textquotesingle{}neat\textquotesingle{} code}
\DocumentationTok{\#\# of the institution types}

\DocumentationTok{\#\# This entire code snippet is tidying up the type of organisation for}
\DocumentationTok{\#\# the 2019 survey renaming the bloody variables to a more java like}
\DocumentationTok{\#\# name}

\NormalTok{dataCyberSecuritySurvey2019TidyName }\OtherTok{=} \FunctionTok{rename}\NormalTok{(dataCyberSecuritySurvey2019,}
    \AttributeTok{instituitionTypes =} \StringTok{"samptype"}\NormalTok{)}

\NormalTok{dataCyberSecuritySurvey2019TidyName}\SpecialCharTok{$}\NormalTok{instituitionTypes }\OtherTok{=} \FunctionTok{as.integer}\NormalTok{(dataCyberSecuritySurvey2019TidyName}\SpecialCharTok{$}\NormalTok{instituitionTypes)}

\FunctionTok{str}\NormalTok{(dataCyberSecuritySurvey2019TidyName}\SpecialCharTok{$}\NormalTok{instituitionTypes)}
\end{Highlighting}
\end{Shaded}

\begin{verbatim}
 int [1:2080] 1 1 1 1 1 1 1 1 1 1 ...
\end{verbatim}

\begin{Shaded}
\begin{Highlighting}[]
\DocumentationTok{\#\# typex is redundant be we already have an indentifies for each type}
\DocumentationTok{\#\# of institution and can be removed same for questtype since this}
\DocumentationTok{\#\# questioner has more redundancy than amazon and google data centers}
\DocumentationTok{\#\# combined}

\NormalTok{dataCyberSecuritySurvey2019TidyName }\OtherTok{=}\NormalTok{ dataCyberSecuritySurvey2019TidyName }\SpecialCharTok{\%\textgreater{}\%}
    \FunctionTok{select}\NormalTok{(}\SpecialCharTok{{-}}\FunctionTok{one\_of}\NormalTok{(}\StringTok{"typex"}\NormalTok{, }\StringTok{"questtype"}\NormalTok{))}
\end{Highlighting}
\end{Shaded}

\begin{Shaded}
\begin{Highlighting}[]
\DocumentationTok{\#\# I continue to save the planet by recycling as much as I can, mostly}
\DocumentationTok{\#\# recycled code from the previous snippet today though this time we do}
\DocumentationTok{\#\# have the concept of education institutions as our code just annoy me}
\DocumentationTok{\#\# after I thought they should be converted to boolean like a getter in}
\DocumentationTok{\#\# java}

\NormalTok{dataCyberSecuritySurvey2020TidyName }\OtherTok{=} \FunctionTok{rename}\NormalTok{(dataCyberSecuritySurvey2020,}
    \AttributeTok{instituitionTypes =} \StringTok{"samptype"}\NormalTok{)}

\NormalTok{dataCyberSecuritySurvey2020TidyName}\SpecialCharTok{$}\NormalTok{instituitionTypes }\OtherTok{=} \FunctionTok{as.integer}\NormalTok{(dataCyberSecuritySurvey2020TidyName}\SpecialCharTok{$}\NormalTok{instituitionTypes)}

\FunctionTok{str}\NormalTok{(dataCyberSecuritySurvey2020TidyName}\SpecialCharTok{$}\NormalTok{instituitionTypes)}
\end{Highlighting}
\end{Shaded}

\begin{verbatim}
 int [1:1900] 1 1 1 1 1 1 1 1 1 1 ...
\end{verbatim}

\begin{Shaded}
\begin{Highlighting}[]
\DocumentationTok{\#\# typex is redundant be we already have an indentifies for each type}
\DocumentationTok{\#\# of institution and can be removed same for questtype since this}
\DocumentationTok{\#\# questioner has more redundancy than amazon and google data centers}
\DocumentationTok{\#\# combined}

\NormalTok{dataCyberSecuritySurvey2020TidyName }\OtherTok{=}\NormalTok{ dataCyberSecuritySurvey2020TidyName }\SpecialCharTok{\%\textgreater{}\%}
    \FunctionTok{select}\NormalTok{(}\SpecialCharTok{{-}}\FunctionTok{one\_of}\NormalTok{(}\StringTok{"typex"}\NormalTok{, }\StringTok{"questtype"}\NormalTok{))}
\end{Highlighting}
\end{Shaded}

\begin{Shaded}
\begin{Highlighting}[]
\DocumentationTok{\#\# saving the planet one recycled snippet of code at a time}


\NormalTok{dataCyberSecuritySurvey2021TidyName }\OtherTok{=} \FunctionTok{rename}\NormalTok{(dataCyberSecuritySurvey2021,}
    \AttributeTok{instituitionTypes =} \StringTok{"samptype"}\NormalTok{)}

\NormalTok{dataCyberSecuritySurvey2021TidyName}\SpecialCharTok{$}\NormalTok{instituitionTypes }\OtherTok{=} \FunctionTok{as.integer}\NormalTok{(dataCyberSecuritySurvey2021TidyName}\SpecialCharTok{$}\NormalTok{instituitionTypes)}

\FunctionTok{str}\NormalTok{(dataCyberSecuritySurvey2021TidyName}\SpecialCharTok{$}\NormalTok{instituitionTypes)}
\end{Highlighting}
\end{Shaded}

\begin{verbatim}
 int [1:1900] 1 1 1 1 1 1 1 1 1 1 ...
\end{verbatim}

\begin{Shaded}
\begin{Highlighting}[]
\DocumentationTok{\#\# typex is redundant be we already have an indentifies for each type}
\DocumentationTok{\#\# of institution and can be removed same for questtype since this}
\DocumentationTok{\#\# questioner has more redundancy than amazon and google data centers}
\DocumentationTok{\#\# combined}

\NormalTok{dataCyberSecuritySurvey2021TidyName }\OtherTok{=}\NormalTok{ dataCyberSecuritySurvey2021TidyName }\SpecialCharTok{\%\textgreater{}\%}
    \FunctionTok{select}\NormalTok{(}\SpecialCharTok{{-}}\FunctionTok{one\_of}\NormalTok{(}\StringTok{"typex"}\NormalTok{, }\StringTok{"questtype"}\NormalTok{))}
\end{Highlighting}
\end{Shaded}

\begin{Shaded}
\begin{Highlighting}[]
\DocumentationTok{\#\# this comment was already dealt by the garbage collector unlike the}
\DocumentationTok{\#\# previous ones}

\NormalTok{dataCyberSecuritySurvey2022TidyName }\OtherTok{=} \FunctionTok{rename}\NormalTok{(dataCyberSecuritySurvey2022,}
    \AttributeTok{instituitionTypes =} \StringTok{"samptype"}\NormalTok{)}

\NormalTok{dataCyberSecuritySurvey2022TidyName}\SpecialCharTok{$}\NormalTok{instituitionTypes }\OtherTok{=} \FunctionTok{as.integer}\NormalTok{(dataCyberSecuritySurvey2022TidyName}\SpecialCharTok{$}\NormalTok{instituitionTypes)}

\FunctionTok{str}\NormalTok{(dataCyberSecuritySurvey2022TidyName}\SpecialCharTok{$}\NormalTok{instituitionTypes)}
\end{Highlighting}
\end{Shaded}

\begin{verbatim}
 int [1:2157] 1 1 1 1 1 1 1 1 1 1 ...
\end{verbatim}

\begin{Shaded}
\begin{Highlighting}[]
\DocumentationTok{\#\# questtype is redundant be we already have an indentifies for each}
\DocumentationTok{\#\# type of institution and can be removed}

\NormalTok{dataCyberSecuritySurvey2022TidyName }\OtherTok{=}\NormalTok{ dataCyberSecuritySurvey2022TidyName }\SpecialCharTok{\%\textgreater{}\%}
    \FunctionTok{select}\NormalTok{(}\SpecialCharTok{{-}}\NormalTok{questtype)}
\end{Highlighting}
\end{Shaded}

\begin{Shaded}
\begin{Highlighting}[]
\DecValTok{1} \SpecialCharTok{+} \DecValTok{1}
\end{Highlighting}
\end{Shaded}

\begin{verbatim}
[1] 2
\end{verbatim}

\begin{Shaded}
\begin{Highlighting}[]
\DecValTok{1} \SpecialCharTok{+} \DecValTok{1}
\end{Highlighting}
\end{Shaded}

\begin{verbatim}
[1] 2
\end{verbatim}

\begin{Shaded}
\begin{Highlighting}[]
\DecValTok{1} \SpecialCharTok{+} \DecValTok{1}
\end{Highlighting}
\end{Shaded}

\begin{verbatim}
[1] 2
\end{verbatim}

\begin{Shaded}
\begin{Highlighting}[]
\DecValTok{1} \SpecialCharTok{+} \DecValTok{1}
\end{Highlighting}
\end{Shaded}

\begin{verbatim}
[1] 2
\end{verbatim}

\begin{Shaded}
\begin{Highlighting}[]
\DecValTok{1} \SpecialCharTok{+} \DecValTok{1}
\end{Highlighting}
\end{Shaded}

\begin{verbatim}
[1] 2
\end{verbatim}

\begin{Shaded}
\begin{Highlighting}[]
\DecValTok{1} \SpecialCharTok{+} \DecValTok{1}
\end{Highlighting}
\end{Shaded}

\begin{verbatim}
[1] 2
\end{verbatim}

\begin{Shaded}
\begin{Highlighting}[]
\DecValTok{1} \SpecialCharTok{+} \DecValTok{1}
\end{Highlighting}
\end{Shaded}

\begin{verbatim}
[1] 2
\end{verbatim}

\begin{Shaded}
\begin{Highlighting}[]
\DecValTok{1} \SpecialCharTok{+} \DecValTok{1}
\end{Highlighting}
\end{Shaded}

\begin{verbatim}
[1] 2
\end{verbatim}

\hypertarget{quarto}{%
\section{Quarto}\label{quarto}}

Quarto enables you to weave together content and executable code into a
finished document. To learn more about Quarto see
\url{https://quarto.org}.

\hypertarget{running-code}{%
\section{Running Code}\label{running-code}}

When you click the \textbf{Render} button a document will be generated
that includes both content and the output of embedded code. You can
embed code like this:

\begin{Shaded}
\begin{Highlighting}[]
\DecValTok{1} \SpecialCharTok{+} \DecValTok{1}
\end{Highlighting}
\end{Shaded}

\begin{verbatim}
[1] 2
\end{verbatim}

You can add options to executable code like this

\begin{verbatim}
[1] 4
\end{verbatim}

The \texttt{echo:\ false} option disables the printing of code (only
output is displayed).



\end{document}
